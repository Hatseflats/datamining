\documentclass[a4paper,12pt]{scrartcl}
\usepackage{graphicx}
\usepackage[none]{hyphenat}
\usepackage{pgfplots}
\def\labelitemi{--}

\pgfplotsset{compat=1.9} 

\begin{document}
\title{Classification Trees}
\subtitle{Data Mining 2015: assignment 1}
\author{Sebastiaan Jong (5546303) \& Bas Geerts (5568978)}
\date{}
\maketitle
\noindent
This report is written for the first assignment of the Data Mining (2015) course at Utrecht University. The goal of this assignment was to write a function in the R programming language that constructs a classification tree on a certain dataset, and to figure out efficient parameters for this tree.
\\
For this assignment we used the Heart Disease dataset from the University of California, Irvine machine learning repository. The preprocessed version for this assignment contains 297 instances and 14 attributes. A brief description of the used attributes:
\begin{enumerate} \itemsep0pt
	\renewcommand\labelitemi{--}
	\item Age, numeric attribute.
	\item Sex, categorical attribute.
	\item ChestPain, categorical attribute.
	\item RestBP, numeric attribute, resting bloodpressure.
	\item Chol, numeric attribute, serum cholesterol.
	\item Fbs, categorical attribute, fasting bloodsugar test result. 
	\item RestECG, categorical attribute, resting electrocardiographic results.
	\item MaxHR, numeric attribute, maximum heart rate achieved.
	\item ExAng, categorical attribute, exercise induced angina. 
	\item Oldpeak, numeric attribute, exercise induced ST-depression. 
	\item Slope, categorical attribute, slope of the peak exercise ST segment.
	\item Ca, numeric attribute, number of major vessels colored by flourosopy.
	\item Thal, categorical attribute, thallium heart scan.
	\item AHD, the class label for this assignment, heart disease diagnosis. 
\end{enumerate}
\clearpage





\end{document}